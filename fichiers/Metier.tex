\PassOptionsToPackage{table}{xcolor}
\documentclass[a4paper, 12pt,french,oneside]{book}%
\usepackage[left=2.6cm,right=1.8cm,top=3cm,bottom=3cm]{geometry}
\usepackage{minitoc}
\usepackage{lettrine}
\usepackage{amsmath,amsthm,amssymb}
\usepackage{glossaries-extra}
\usepackage{glossary-longbooktabs}
\usepackage{glossaries-extra-stylemods}
\usepackage{systeme,mathtools}
\usepackage[utf8]{inputenc}
\usepackage[T1]{fontenc}
\usepackage{listings}
\usepackage{babel}
\usepackage{fourier, erewhon, cabin}
\usepackage{microtype}
\usepackage{float}
\usepackage{array,multirow}
\usepackage{physics}
\usepackage{subfig}
\usepackage[pagestyles, clearempty, explicit]{titlesec}
\usepackage{eqparbox}
\usepackage{graphicx}
\usepackage[section]{placeins}
\usepackage[dvipsnames]{xcolor}
\usepackage{booktabs,siunitx}
\usepackage{colortbl}
\usepackage{makecell}
\usepackage{eso-pic}
\usepackage{fancyhdr}
\usepackage{caption}
\usepackage{wrapfig}
\usepackage{engrec}
\usepackage{relsize}
\usepackage{microtype}
\usepackage[most]{tcolorbox}
\usepackage[toc,page]{appendix}
\usepackage{epigraph}
\usepackage{times}
\usepackage{tikz}
\usepackage[pages=some]{background}
\usepackage{epigraph}
\pagestyle{fancy}
\setcounter{tocdepth}{3}


\doparttoc
\fancyhf{}
\fancyhead[CE,CO]{\leftmark}
\fancyhead[LE,RO]{\thepage}
\renewcommand{\headrulewidth}{1pt}
\addto\captionsfrench{\def\tablename{Tableau}}
\definecolor{lightgray}{gray}{0.95}
\setlength{\arrayrulewidth}{0.5mm}
\setlength{\tabcolsep}{5pt}
\renewcommand{\arraystretch}{1.5}
\renewcommand{\baselinestretch}{1.46}
\renewcommand{\appendixpagename}{Annexes}
\renewcommand{\appendixtocname}{Annexes}
\renewcommand{\bibliography}{Referances}
\renewcommand{\thesubsubsection}{\alph{subsubsection}}
\titleformat{\chapter}[block]{}{\eqmakebox[chap]{\Large\MakeUppercase{\sffamily\lsstyle\chaptername} %
		\raisebox{-0.6\height}{\fontsize{50pt}{50pt}\selectfont\thechapter}\quad}}%
{0pt}{\huge\bfseries\raisebox{1ex}{\parbox[t]{\dimexpr\textwidth-\eqboxwidth{chap}\relax}{\titlerule[2pt]\vspace{1.25ex}#1}}}
\titlespacing*{\chapter}{0pt}{-32pt}{48pt}%
\setcounter{secnumdepth}{4}
\setcounter{tocdepth}{4}
\theoremstyle{definition}
\newtheorem{definition}{Définition}[chapter]
\theoremstyle{remark}
\newtheorem{remark}{\textbf{Remarque}}[chapter]
\usepackage{times}
\addto\captionsfrench{\renewcommand{\chaptername}{partie}}
\setlength{\textfloatsep}{2pt} 
\setlength{\abovecaptionskip}{2pt}
\setlength{\belowcaptionskip}{2pt}
\setlength{\floatsep}{2pt}
\setlength{\dblfloatsep}{2pt}
\setlength{\dbltextfloatsep}{2pt} 
\setlength{\leftmargini}{8pt}
\setlength{\leftmarginii}{8pt}
\setlength{\columnsep}{9pt}
\setlength{\parskip}{0pt plus0pt minus1pt}
\setlength\epigraphrule{1pt}
\setlength\epigraphwidth{.5\textwidth}
\renewcommand\epigraphflush{flushleft}
\renewcommand\epigraphsize{\normalsize}
\setlength\epigraphwidth{0.7\textwidth}
\definecolor{ch}{rgb}{255,255,255}
\definecolor{A}{rgb}{0.25, 0.29, 0.3}
\definecolor{C}{cmyk}{10,0,0,0}
\definecolor{M}{rgb}{0.9, 0.9, 0.98}
\definecolor{N}{rgb}{0.8, 0.8, 1.0}
\DeclareFixedFont{\titlefont}{T1}{ppl}{b}{it}{0.5in}
\newcommand\titlepagedecoration[1]{%
	\begin{tikzpicture}[remember picture,overlay,shorten >= -10pt]
	\coordinate (tp1) at ([yshift=2cm]current page.west);
	\coordinate (tp2) at ([yshift=2cm,xshift=17.38cm]current page.west);
	\coordinate (tp3) at ([yshift=-90pt,xshift=6.95cm]current page.north);
	\coordinate (tp4) at ([yshift=-90pt]current page.north west);
	
	\filldraw[draw=ch,fill=ch] (tp1)--(tp2)--(tp3)--(tp4)--cycle;
	\filldraw[draw=ch!30!white,opacity=0] ([xshift=-5cm]tp1)--([xshift=-1cm]tp2)--([xshift=-1cm]tp3)--([xshift=-5cm]tp4)--cycle;
	\node[right] at ([xshift=1cm,yshift=-8cm]current page.north west) {\parbox{\textwidth}{\color{white}#1}};
	\end{tikzpicture}%
}
\usepackage{tikz}
\definecolor{doc}{RGB}{0,60,110}
\usepackage{titletoc}
\contentsmargin{0cm}



%\titlecontents{section}[2.4pc]
%{\addvspace{1pt}}
%{\contentslabel[\thecontentslabel]{2pc}}
%{}
%{\hfill\small \thecontentspage}
%[]
%\titlecontents{subsection}[4pc]
%{\addvspace{1pt}}
%{\contentslabel[\thecontentslabel]{2.4pc}}
%{}
%{\hfill\small \thecontentspage}
%[]
%\titlecontents{subsubsection}[6pc]
%{\addvspace{1pt}}
%{\contentslabel[\thecontentslabel]{1.5pc}}
%{}
%{\hfill\small \thecontentspage}
%[]



\makeatletter

\makeatother
%%Background%%%%%%%%%%%%%%%%%%%%
\backgroundsetup{
	scale=1,
	color=black,
	opacity=1,
	angle=0,
	contents={%
	%	\includegraphics[width=\paperwidth,height=\paperheight]{image/page}
	}%
}
\dominitoc

\begin{document}
	\thispagestyle{empty}
\begin{titlepage}
	\begin{center}
		\bfseries
		\begin{minipage}{0.2\linewidth}
			\raggedright
			\includegraphics[scale=.11]{image/ensias}%
		\end{minipage}
		\hfill
		\begin{minipage}{0.2\linewidth}
			\raggedleft
			\includegraphics[scale=.5]{image/um5}%
		\end{minipage}
		\vskip.3in
		\rule{\textwidth}{1.5pt}
		\vskip-.04in
		\textsc{\Large Université Mohammed V Rabat}
		\vskip.05in
		\textsc{\large École nationale supérieure d'informatique et d'analyse des systèmes}
		\vskip.01in
		\vskip-.1in
		\rule{\textwidth}{1.5pt}
		\vskip0.63in
		\large Génie logiciel
		\vskip0.1in
		\rule{\textwidth}{4pt}
		\vskip-0.27in
		\rule{\textwidth}{1.5pt}
		\vskip-0.03in
		\vskip0.02in
		\Huge Bilan Personnel
		\vskip-0.19in
		\rule{\textwidth}{1.5pt}
		\vskip-0.44in
		\rule{\textwidth}{4pt}
		\vskip0.2in
		\emph{\Large Encadré par Pr. Soumia EZZAHID }
	\end{center}
	\vskip0.5in
	
	\begin{minipage}{.5\textwidth}
		\begin{flushleft}
			\bfseries \textcolor{MidnightBlue}{\large Réalisé par :}\par \emph{Ismail BARKANI}
		\end{flushleft}
	\end{minipage}
\hskip.1\textwidth

	\vskip1in
	\centering
	\bfseries
\end{titlepage}
\restoregeometry 

\restoregeometry
\dominitoc% Initialization
\setcounter{tocdepth}{7}
\tableofcontents
\thispagestyle{empty}
\chapter{Bilan personnel}
\section{A propos }
Je suis Ismail Barkani, j'ai 23 ans, d'origine Nador, actuellement je suis un étudiant en deuxième année filière GL à l'école national supérieur d'informatique et d'analyse des système.

J'ai plusieurs qualités telles que l'honnête et l'adaptation. Au fur et à mesure des années j’ai donc appris à ne plus dire tout ce que je pensais et à être diplomate. Pour l'adaptation, je m'adapte très facilement et très rapidement à un nouvel environnement. Grâce à mes études dans des villes différentes (Oujda, Tanger, Rabat), je sais à qui m'adresser si j'ai des difficultés ou si j'ai des questions à poser.

La procrastination, c'est l'une de mes défauts, je remets souvent au lendemain ce que je peux faire le jour même. Je travaille fréquemment à la dernière minute.
\section{Parcours universitaire}
Actuellement je suis un étudiant en génie informatique, mais avant cela, j'ai parcouru un long chemin dans le monde informatique et mathématique.

Après avoir obtenue mon baccalauréat science mathématiques A en 2014, je me suis dirigé vers la faculté de science d'Oujda afin d'obtenir un licence en science de mathématiques et d'application en 2017 avec un mention très bien et j'étais le majorant de ma promotion. Cet honneur m'a permis d'intégrer l'ENSIAS en 2018.
\section{Expériences}
Le juillet 2011, je commençais ma première expérience freelance comme yn 'Graphic Designer'.Après beaucoup de questionnements existentiels, j’avais décidé de tenter l’aventure pour devenir travailleuse indépendante, donc j'étais un freelancer pendant 3 ans, en 2014 j'arrête de faire le freelance pour se concentrer sur mes études.

J'avais effectué un stage de deux mois au sein de LogimaStore à Oujda, j’ai acquis de vraies compétences dans mon domaine informatique, et j'ai eu la chance de travailler avec les nouvelles technologies.

 part ça, je suis actullement le responsable de  cellule design dans un club à l'ENSIAS, qui s'appelle HULT Prize, et aussi le responsable techniques ches MCSC (Moroccan cyber security camp) .
\section{Ce que j'aime de faire}
J'ai commencé à aimer le domaine technologique avec les jeux vidéo classiques, puis j'ai commencé la programmation à 14 ans (langage C \& C++). Depuis chaque jour, je me sens chanceux et reconnaissant d'avoir choisi cette voie et d'essayer de repousser encore plus mes limites. Pour moi, le code est de la poésie et la chasse aux bogues est un jeu d'échecs contre moi-même.
\section{Ce que je voudrais être}
Dans 3 ans, je me vois travailler au sein d'un startup . Je m'imagine d'être un développeur fullStack. Mon lieu de vie rêvé dans 3 ans est la ville de Grenoble à la france.\\

Dans 10 ans, je me vois un Devops Senior dans un grand boite. 
Je me projette dans 10 ans et je me vois avec une vie  stable, marié, j'ai une belle maison à la campagne.

\end{document}\\
	