\PassOptionsToPackage{table}{xcolor}
\documentclass[a4paper, 12pt,french,oneside]{book}%
\usepackage[left=2.6cm,right=1.8cm,top=3cm,bottom=3cm]{geometry}
\usepackage{minitoc}
\usepackage{lettrine}
\usepackage{amsmath,amsthm,amssymb}
\usepackage{glossaries-extra}
\usepackage{glossary-longbooktabs}
\usepackage{glossaries-extra-stylemods}
\usepackage{systeme,mathtools}
\usepackage[utf8]{inputenc}
\usepackage[T1]{fontenc}
\usepackage{listings}
\usepackage{babel}
\usepackage{fourier, erewhon, cabin}
\usepackage{microtype}
\usepackage{float}
\usepackage{array,multirow}
\usepackage{physics}
\usepackage{subfig}
\usepackage[pagestyles, clearempty, explicit]{titlesec}
\usepackage{eqparbox}
\usepackage{graphicx}
\usepackage[section]{placeins}
\usepackage[dvipsnames]{xcolor}
\usepackage{booktabs,siunitx}
\usepackage{colortbl}
\usepackage{makecell}
\usepackage{eso-pic}
\usepackage{fancyhdr}
\usepackage{caption}
\usepackage{wrapfig}
\usepackage{engrec}
\usepackage{relsize}
\usepackage{microtype}
\usepackage[most]{tcolorbox}
\usepackage[toc,page]{appendix}
\usepackage{epigraph}
\usepackage{times}
\usepackage{tikz}
\usepackage[pages=some]{background}
\usepackage{epigraph}
\pagestyle{fancy}
\setcounter{tocdepth}{3}


\doparttoc
\fancyhf{}
\fancyhead[CE,CO]{\leftmark}
\fancyhead[LE,RO]{\thepage}
\renewcommand{\headrulewidth}{1pt}
\addto\captionsfrench{\def\tablename{Tableau}}
\definecolor{lightgray}{gray}{0.95}
\setlength{\arrayrulewidth}{0.5mm}
\setlength{\tabcolsep}{5pt}
\renewcommand{\arraystretch}{1.5}
\renewcommand{\baselinestretch}{1.46}
\renewcommand{\appendixpagename}{Annexes}
\renewcommand{\appendixtocname}{Annexes}
\renewcommand{\bibliography}{Referances}
\renewcommand{\thesubsubsection}{\alph{subsubsection}}
\titleformat{\chapter}[block]{}{\eqmakebox[chap]{\Large\MakeUppercase{\sffamily\lsstyle\chaptername} %
		\raisebox{-0.6\height}{\fontsize{50pt}{50pt}\selectfont\thechapter}\quad}}%
{0pt}{\huge\bfseries\raisebox{1ex}{\parbox[t]{\dimexpr\textwidth-\eqboxwidth{chap}\relax}{\titlerule[2pt]\vspace{1.25ex}#1}}}
\titlespacing*{\chapter}{0pt}{-32pt}{48pt}%
\setcounter{secnumdepth}{4}
\setcounter{tocdepth}{4}
\theoremstyle{definition}
\newtheorem{definition}{Définition}[chapter]
\theoremstyle{remark}
\newtheorem{remark}{\textbf{Remarque}}[chapter]
\usepackage{times}
\addto\captionsfrench{\renewcommand{\chaptername}{partie}}
\setlength{\textfloatsep}{2pt} 
\setlength{\abovecaptionskip}{2pt}
\setlength{\belowcaptionskip}{2pt}
\setlength{\floatsep}{2pt}
\setlength{\dblfloatsep}{2pt}
\setlength{\dbltextfloatsep}{2pt} 
\setlength{\leftmargini}{8pt}
\setlength{\leftmarginii}{8pt}
\setlength{\columnsep}{9pt}
\setlength{\parskip}{0pt plus0pt minus1pt}
\setlength\epigraphrule{1pt}
\setlength\epigraphwidth{.5\textwidth}
\renewcommand\epigraphflush{flushleft}
\renewcommand\epigraphsize{\normalsize}
\setlength\epigraphwidth{0.7\textwidth}
\definecolor{ch}{rgb}{255,255,255}
\definecolor{A}{rgb}{0.25, 0.29, 0.3}
\definecolor{C}{cmyk}{10,0,0,0}
\definecolor{M}{rgb}{0.9, 0.9, 0.98}
\definecolor{N}{rgb}{0.8, 0.8, 1.0}
\DeclareFixedFont{\titlefont}{T1}{ppl}{b}{it}{0.5in}
\newcommand\titlepagedecoration[1]{%
	\begin{tikzpicture}[remember picture,overlay,shorten >= -10pt]
	\coordinate (tp1) at ([yshift=2cm]current page.west);
	\coordinate (tp2) at ([yshift=2cm,xshift=17.38cm]current page.west);
	\coordinate (tp3) at ([yshift=-90pt,xshift=6.95cm]current page.north);
	\coordinate (tp4) at ([yshift=-90pt]current page.north west);
	
	\filldraw[draw=ch,fill=ch] (tp1)--(tp2)--(tp3)--(tp4)--cycle;
	\filldraw[draw=ch!30!white,opacity=0] ([xshift=-5cm]tp1)--([xshift=-1cm]tp2)--([xshift=-1cm]tp3)--([xshift=-5cm]tp4)--cycle;
	\node[right] at ([xshift=1cm,yshift=-8cm]current page.north west) {\parbox{\textwidth}{\color{white}#1}};
	\end{tikzpicture}%
}
\usepackage{tikz}
\definecolor{doc}{RGB}{0,60,110}
\usepackage{titletoc}
\contentsmargin{0cm}



%\titlecontents{section}[2.4pc]
%{\addvspace{1pt}}
%{\contentslabel[\thecontentslabel]{2pc}}
%{}
%{\hfill\small \thecontentspage}
%[]
%\titlecontents{subsection}[4pc]
%{\addvspace{1pt}}
%{\contentslabel[\thecontentslabel]{2.4pc}}
%{}
%{\hfill\small \thecontentspage}
%[]
%\titlecontents{subsubsection}[6pc]
%{\addvspace{1pt}}
%{\contentslabel[\thecontentslabel]{1.5pc}}
%{}
%{\hfill\small \thecontentspage}
%[]



\makeatletter

\makeatother
%%Background%%%%%%%%%%%%%%%%%%%%
\backgroundsetup{
	scale=1,
	color=black,
	opacity=1,
	angle=0,
	contents={%
	%	\includegraphics[width=\paperwidth,height=\paperheight]{image/page}
	}%
}
\dominitoc

\begin{document}
	\thispagestyle{empty}
\begin{titlepage}
	\begin{center}
		\bfseries
		\begin{minipage}{0.2\linewidth}
			\raggedright
			\includegraphics[scale=.11]{image/ensias}%
		\end{minipage}
		\hfill
		\begin{minipage}{0.2\linewidth}
			\raggedleft
			\includegraphics[scale=.5]{image/um5}%
		\end{minipage}
		\vskip.3in
		\rule{\textwidth}{1.5pt}
		\vskip-.04in
		\textsc{\Large Université Mohammed V Rabat}
		\vskip.05in
		\textsc{\large École nationale supérieure d'informatique et d'analyse des systèmes}
		\vskip.01in
		\vskip-.1in
		\rule{\textwidth}{1.5pt}
		\vskip0.63in
		\large Génie logiciel
		\vskip0.1in
		\rule{\textwidth}{4pt}
		\vskip-0.27in
		\rule{\textwidth}{1.5pt}
		\vskip-0.03in
		\vskip0.02in
		\Huge Prospection du marché
		\vskip-0.19in
		\rule{\textwidth}{1.5pt}
		\vskip-0.44in
		\rule{\textwidth}{4pt}
		\vskip0.2in
		\emph{\Large Encadré par Pr. Soumia EZZAHID }
	\end{center}
	\vskip0.5in
	
	\begin{minipage}{.5\textwidth}
		\begin{flushleft}
			\bfseries \textcolor{MidnightBlue}{\large Réalisé par :}\par \emph{Ismail BARKANI}
		\end{flushleft}
	\end{minipage}
\hskip.1\textwidth

	\vskip1in
	\centering
	\bfseries
\end{titlepage}
\restoregeometry 

\restoregeometry
\dominitoc% Initialization
\setcounter{tocdepth}{7}
\tableofcontents
\thispagestyle{empty}

\chapter{Fiche métier:\textit{Développeur FullStack}}
Cette partie est consacrée à la présentation de la métier qui m'intéresse dans le secteur informatique. 
\section{Autres intitulés}
Aucun car ce métier est une combinaison de plusieurs métiers (développeur front-end, backend, web architect... )
\section{Type d’employeurs et secteurs d’activité}
Annonceur, E-commerce, Edition / Médias / Régies, Freelance, Pure player, Start-up
\section{Missions}
Le développeur full-stack a pour principale mission la programmation d’une application ou
d’un site : il est à même de le concevoir de A à Z (création, développement, codage, etc.). Il a des compétences dans tous les domaines : back-end, front-end, UX/UI, architecture...
Ce profil est donc très prisé des start-ups, où le nombre d’employés est limité et où un profil polyvalent est particulièrement apprécié.

Ce profil généraliste permet donc d’intervenir sur des missions diversifiées et à différents niveaux dans la conception du site, soit en globalité, soit sur certains aspects précis, tout dépend des besoins de l’entreprise. Il peut jongler entre plusieurs missions faisant intervenir différents outils de programmation au cours d’une même journée.

Le développeur full-stack est passionné de code, il a des compétences informatiques poussées ainsi qu’une parfaite maîtrise des langages de programmation et des bases de données. Il maitrise également les notions d’API pour dialoguer avec des sites partenaires
\section{Évolution de carrière}
Le développeur full-stack peut évoluer en tant que CTO, Directeur de site, Lead développeur,
Directeur de produit, Head of digital... Ses nombreuses compétences techniques et son ouverture sur la partie business et marketing lui permettent d’exercer des postes variés.
\section{Salaire d’un Développeur full-stack} 
Le salaire varie beaucoup en fonction de la taille l’entreprise.\\
\textbf{Junior :} 8000Dh-11000Dh\\
\textbf{Senior :} 12000Dh-25000Dh
\section{Formations privilégiées}
Les formation privilégiées sont :\\
$\bullet$ École d’ingénieur en génie informatique.\\
$\bullet$ Master en informatique.
\section{Les compétences du développeur full-stack}
Le développeur full-stack doit maîtriser les principales technologies et les principaux langages de programmation actuellement utilisés s’il veut pouvoir à la fois intervenir sur le front end et le back end des sites Internet ou des applications. 

Ce professionnel doit donc posséder des compétences généralistes. Il doit aussi savoir faire preuve d’adaptation, car dans la même journée, il peut être amené à travailler sur des missions vraiment très différentes les unes des autres.
\section{Les connaisances du développeur full-stack}
principalement :\\
\textbf{Back-End :} JEE (sesframworks : Spring, Jsf, Scruts...), Nodejs, python (et sesframworks : Django...), php (et ces framworks : Laravel, Symfony)\\
\textbf{Front-End : }Html, Css, Bootstrap,Sass, JavaScript(et
ces framworks : Angular, React, Vuejs...)
\chapter{Carte d'identité d'entreprise: \textit{Capgemini}}
\begin{minipage}{\linewidth}
	\centering
		\includegraphics[keepaspectratio=true,scale=0.2]{image/cap}
\end{minipage}\vspace{0.3cm}\\

\section{bref présentation}
Voici une bref présentation sur \textbf{\textit{Capgemini}}:\\
\textbf{$\rhd$ Création:} 1er octobre 1967\\
\textbf{$\rhd$ Fondateurs:} Serge Kampf\\
\textbf{$\rhd$ Forme juridique:} Société européenne 1 à conseil d'administration\\
\textbf{$\rhd$ Action:} Euronext : CAP\\
\textbf{$\rhd$ Slogan:} ’’\textit{Homme est vital, le résultat capital}’’.\\
\textbf{$\rhd$ Siège social:} 11 rue de Tilsitt, 75017 Paris, France\\
\textbf{$\rhd$ Direction:} Paul Hermelin, président-directeur général\\
\textbf{$\rhd$ Activité:} \textbf{ESN}( Entreprise de services du numérique)\\
\textbf{$\rhd$ Produits:} Conseil en stratégie et transformation, Services applicatifs, Services de technologie et d'ingénierie, Autres services d'infogérance \\
\textbf{$\rhd$ Filiales:} Sogeti, Capgemini Invent\\
\textbf{$\rhd$ Effectif:} 	211 300\\
\section{présentation}
\textbf{Capgemini} est la première entreprise de services du numérique (ESN) en France ainsi que le numéro six mondial du secteur en 20167,8. Basée à Paris, la société fait partie du CAC 40 à la Bourse de Paris.

Elle a été créée par\textbf{ Serge Kampf }le 1er octobre 1967 à Grenoble (France) sous le nom de Sogeti (Société pour la gestion de l'entreprise et traitement de l'information).
\section{Écosystème}
Le groupe Capgemini en France est constitué de neuf entités juridiques :\\
$\bullet$ Capgemini Consulting\\
$\bullet$ Capgemini France\\
$\bullet$ Capgemini Outsourcing\\
$\bullet$ Capgemini Service\\
$\bullet$ Capgemini Technology Services\\
\end{document}\\
	